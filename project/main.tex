\documentclass[a4paper]{article}

\usepackage[utf8]{inputenc}
\usepackage[spanish]{babel}
\usepackage{geometry}
\geometry{margin=3cm}
\usepackage{anyfontsize}
\usepackage{matlab-prettifier}
\usepackage{amssymb,amsmath,amsfonts}

\begin{document}
\begin{titlepage}
    \begin{center}
        \vspace*{2cm}
        
        \Large
        \textbf{TRABAJO I4} \\[0.5cm]
        \large
        Asignatura: Análisis Numérico \\[2cm]
        
        {\fontsize{40}{48}\selectfont
        \textbf{Integración numérica de Gauss-Lobatto}} \\[3cm]
        
        \Large
        Grupo 2 \\[0.5cm]
        Curso 2024--2025 \\[3cm]
        
        \textbf{Autores:} \\[0.5cm]
        de Juan Lopez, Jesús \\
        León Martín, Jürgen \\
        Navarro Rubio, Pablo \\
        Sánchez Montoya, María Cruz \\[4cm]
        
    \end{center}
\end{titlepage}

\tableofcontents

\newpage


\section{Introducción}
En este trabajo implementaremos y analizaremos la integración numérica por medio de la fórmula de cuadratura de Gauss-Lobatto para n nodos, que sigue la fórmula
\[
\int_{-1}^{1} \! f(t)  \,dt = w_1 f(-1) + w_n f(1) + \sum_{i=1}^{n-1}w_i f(t_i),
\]
donde $t_i$, $i=2, 3,...,n-1$, son las raíces del polinomio $p'_{n-1}(t)$, siendo $p_{n-1}(t)$ el polinomio de Legendre de grado $n-1$. Los pesos son $w_1 = w_n = \tfrac{2}{n(n-1)}$ y
\[
w_i=\frac{2}{n(n-1)(p_{n-1}(t_i))^2},\qquad i=2,3,...,n-1.
\]
Y, finalmente, compararemos los resultados obtenidos con otro método similar, Gauss-Legendre.


\section{Construcción del programa de Gauss-Lobatto}
El programa que queremos obtener debe tener como parámetro de entrada el número de nodos \textit{n}, y de salida el vector de nodos \textit{ti} y el de pesos \textit{wi}.

El orden que debemos seguir para el programa es:
\begin{enumerate}
\item Evaluar los posibles casos según los nodos pedidos.
\item El vector de nodos intermedios \textit{ti}: Polinomio de Legendre de grado $n-1$, su derivada y las raíces de esta.
\item El vector de pesos \textit{wi}.
\item El vector de nodos completo \textit{ti} y programa final.
\end{enumerate}

\subsection{Posibles casos según los nodos pedidos}
Hay que tener en cuenta tres posibles casos:
\begin{enumerate}
\item \textbf{\boldmath Si $n<2$}

En este caso, no es posible calcular la cuadratura por varias razones:

    \begin{itemize}
    \item \textbf{\boldmath Si $n=1$}
        \begin{itemize}
        \item No hay suficientes nodos para incluir los extremos de la cuadratura $-1$ y 1.
        \item Con un único nodo no captaría ninguna curvatura, luego no hay forma de aproximar con precisión.
        \end{itemize}
    \item \textbf{\boldmath Si $n<1$}
        \begin{itemize}
        \item No tiene sentido calcular una cuadratura sin nodos, o con nodos negativos.
        \end{itemize}
    \end{itemize}

En este caso, el programa mostrará el mensaje \textit{"Se necesitan al menos 2 nodos para la cuadratura"}.

\item \textbf{\boldmath Si $n=2$}

Si se piden únicamente 2 nodos, el vector de nodos $t_i$ estará formado por los extremos de la cuadratura $-1$ y 1. Entonces, solo sería necesario calcular los pesos de los extremos $w_1$ y $w_n$ para obtener el vector de pesos $wi$.

\item \textbf{\boldmath Si $n>2$}

En este caso, habría que calcular los nodos intermedios por medio de encontrar las raíces de la derivada del polinomio de Legendre de grado $n-1$. Tras esto, ya tendríamos el vector de nodos $t_i$. Solo faltaría obtener los pesos de los extremos $w_1$ y $w_n$, y los intermedios $w_i$ para conseguir el vector de pesos \textit{wi}. 

\end{enumerate} 

\subsection{El vector de nodos intermedios \textit{ti}}
\subsubsection{\boldmath Polinomio de Legendre de orden $n-1$}
En este primer paso, usamos un código auxiliar \textit{LegendreT.m} que devuelve los coeficientes truncados del polinomio de Legendre del orden que le introduzcamos, en este caso, $n-1$. Estos coeficientes (dispuestos de forma descendente) los guardamos en el vector \textit{cl}.
%Por qué usamos Legendre truncado?

\textit{\\Código:}

\begin{lstlisting}[frame=single, style=Matlab-Pyglike]
cl=LegendreT(n-1);
\end{lstlisting}

\subsubsection{\boldmath Derivada del polinomio de Legendre de orden $n-1$}
Aplicamos la regla básica de derivación de polinomios:
\[
\frac{d}{dx}(a_k x^k)=k a_k x^{k-1}
\]

Multiplicamos componente a componente el vector de coeficientes \textit{cl} por otro vector con los grados de cada término del polinomio, en orden descendente. Así, cada coeficiente se multiplica por su grado, obteniendo los coeficientes de la derivada, que guardamos en el vector \textit{ci}.

Sin embargo, la derivada del polinomio de grado $n-1$ tiene grado $n-2$, por lo que eliminamos el último término correspondiente al grado 0, cuya derivada es 0.

\textit{\\Código:}

\begin{lstlisting}[frame=single, style=Matlab-Pyglike]
ci=((n-1):-1:0).*cl;
ci=ci(:,(1:n-1));
\end{lstlisting}

\subsubsection{\boldmath Raíces de la derivada del polinomio de Legendre de orden $n-1$}
Para calcular los ceros de la derivada del polimonio de Legendre, usamos el comando \textit{roots} que devuelve las raíces de un polinomio en forma de vector columna, al que llamamos \textit{ti}, obteniendo así los nodos intermedios. Como queremos trabajar con vectores fila, trasponemos \textit{ti}. Además, lo ordenamos de forma descendente con el comando \textit{sort}. 

\textit{\\Código:}

\begin{lstlisting}[frame=single, style=Matlab-Pyglike]
ti=roots(ci);
ti=ti';
ti=sort(ti,"descend");
\end{lstlisting}

\subsection{El vector de pesos \textit{wi}}
En primer lugar, calculamos los pesos de los extremos $w_1$ y $w_n$, que asignamos a la variable \textit{wn}, y definimos el vector, por ahora vacío, \textit{wi}.

Luego, calculamos mediante un bucle \textit{for} los pesos intermedios $w_i$, que se pueden expresar como
\[
w_i=w_n\frac{1}{(p_{n-1}(y_i))^2},\qquad i=2,3,...,n-1.
\]
Y los vamos añadiendo al vector \textit{wi}, obteniendo así el vector de pesos intermedios.

Finalmente, completamos el vector \textit{wi} agregándole los pesos de los extremos \textit{wn}.

\textit{\\Código:}

\begin{lstlisting}[frame=single, style=Matlab-Pyglike]
for i=1:(n-2)
    pi=polyval(cl,ti(i));
    w=wn/(pi^2);
    wi=[wi w];
end
wi=[wn wi wn];
\end{lstlisting}

\subsection{El vector de nodos completo \textit{ti} y programa final}
Solo falta añadir los nodos de los extremos $-1$ y 1 al vector \textit{ti}. Con esto, ya tenemos el vector de nodos \textit{ti} completo.

\textit{\\Código:}

\begin{lstlisting}[frame=single, style=Matlab-Pyglike]
ti=[1 ti -1];
\end{lstlisting}

Finalmente, construimos el programa final:

\begin{lstlisting}[frame=single, numbers=left, style=Matlab-Pyglike]
function [ti,wi] = GaussLobatto(n)
%[ti,wi] = GaussLobatto(n)
wi=[];
wn=2/(n*(n-1));
if n>2
    cl=LegendreT(n-1);
    ci=((n-1):-1:0).*cl;
    ci=ci(:,(1:n-1));
    ti=roots(ci);
    ti=ti';
    ti=sort(ti,"descend");
    for i=1:(n-2)
        pi=polyval(cl,ti(i));
        w=wn/(pi^2);
        wi=[wi w];
    end
    wi=[wn wi wn];
    ti=[1 ti -1];
else 
    if n==2
        wi=[wn wn];
        ti=[1 -1];
    else
    disp("Se necesitan al menos 2 nodos para la cuadratura")
    end
end
end
\end{lstlisting}

\newpage
\section{Pesos y nodos de la cuadratura de Gauss-Lobatto para 8 y 9 nodos.}

\subsection{Simetría de los nodos respecto al 0}

\subsubsection{Paridad e imparidad de los polinomios de Legendre}

Queremos probar que el polinomio de Legendre $l_i$ es función par si $i=2k$ e impar si $i=2k+1$, $k\in\mathbb{N}\cup\{0\}$ por el método de inducción. Podemos ver que para los casos $i=0, 1, 2, 3$ se cumple:
\[l_0=1\quad;\quad l_1=x\quad;\quad l_2=x^2-\frac{1}{3}\quad;\quad l_3=x^3-\frac{3}{5}x \]

Supongamos que $l_i$ es par y $l_{i+1}$ impar y probemos para $l_{i+2}$ y $l_{i+3}$ :\\

a) Sabemos que
\[l_{i+2}(x)=(x-b_{i+2})\cdot l_{i+1}(x)-c_{i+2}\cdot l_i(x)\quad, \quad\alpha_{i+1}=\int_{-1}^1l_{i+1}^2(x)dx>0\]

Por lo tanto\[b_{i+2}=\frac{1}{\alpha_{i+1}} \int_{-1}^1\underbrace{x\cdot l_{i+1}^2(x)}_{impar}dx=0\Rightarrow l_{i+2}=\underbrace{x\cdot l_{i+1}(x)}_{par}-\underbrace{c_{i+2}\cdot l_i(x)}_{par}\Rightarrow l_{i+2}(x)\text{ es par.}\]\\

b) Sabemos que
\[l_{i+3}(x)=(x-b_{i+3})\cdot l_{i+2}(x)-c_{i+3}\cdot l_{i+1}(x)\quad,\quad\alpha_{i+2}=\int_{-1}^1l_{i+2}^2(x)dx>0\] 

Por lo tanto \[b_{i+3}=\frac{1}{\alpha_{i+2}}\int_{-1}^1\underbrace{x\cdot l_{i+2}^2(x)}_{impar}dx=0
\Rightarrow b_{i+3}=\underbrace{x\cdot l_{i+2}(x)}_{impar}-\underbrace{c_{i+3}\cdot l_{i+1}(x)}_{impar}\Rightarrow l_{i+3}(x)\text{ es par.}\]

Este resultado también se extrapola a los polinomios de Legendre truncados utilizados en el trabajo, dado que son simplemente un escalado de los polinomios de Legendre normales.

\subsubsection{Simetría de las raíces de funciones pares e impares}

Queremos probar que tanto para las funciones de grado par como las de grado impar, el conjunto de raíces debe ser simétrico entorno al cero.
\begin{enumerate}
\item \textbf{Caso par:}

Supongamos que existe una raíz $x_i:f(x_i)=0$. Al ser $f(x)$ una función par, se cumple que $f(-x_i)=f(x_i)=0$. Por lo tanto, se cumple que para cada raíz de la función par, su elemento opuesto también es raíz.

\item \textbf{Caso impar:}

Partamos de la hipótesis planteada en el caso anterior. Al tratarse de una función impar, se cumple que $-f(x_i)=f(x_i)=0\Rightarrow f(-x_i)=0$. Ocurre lo mismo que en el caso par.
\end{enumerate}
\subsubsection{\boldmath Simetría de las raices de $\tilde l_i(x)'$}

Utilizando el resultado del apartado 3.1.1, vemos que las derivadas de los polinomios de Legendre truncados también deben ser o pares o impares, dado que al derivar el polinomio se invierte la paridad. Por el resultado del apartado 3.1.2 el conjunto de raíces de $\tilde l_i(x)'$ (que corresponden con los nodos de Gauss-Lobatto) debe ser simétrico entorno al 0.

\subsection{Suma de los pesos}

Ambos resultados tienen sentido pues
\begin{itemize}
\item Los vectores de nodos, \textit{t8} y \textit{t9}, son simétricos respecto del 0. %completar
\item Los vectores de pesos, \textit{w8} y \textit{w9}, son simétricos respecto del 0 y suman 2. 
\noindent Para demostrar que la suma de los pesos es siempre 2, consideramos el sistema matricial asociado a la cuadratura de Gauss-Lobatto con \( n \) nodos. Este garantiza la exactitud para polinomios de grado \( \leq 2n-3 \):

\[
\begin{cases}
    \sum_{i=1}^n w_i \cdot 1 = \int_{-1}^1 1 \, dx = 2, & \text{(exactitud para \( f(x) = 1 \))} \\
    \sum_{i=1}^n w_i x_i = \int_{-1}^1 x \, dx = 0, & \text{(exactitud para \( f(x) = x \))} \\
    \quad \vdots & \\
    \sum_{i=1}^n w_i x_i^{2n-3} = \int_{-1}^1 x^{2n-3} \, dx. & \text{(exactitud para \( f(x) = x^{2n-3} \))}
\end{cases}
\]

\noindent En forma matricial, el sistema se escribe como:
\[
\underbrace{
\begin{bmatrix}
1 & 1 & \cdots & 1 \\
x_1 & x_2 & \cdots & x_n \\
x_1^2 & x_2^2 & \cdots & x_n^2 \\
\vdots & \vdots & \ddots & \vdots \\
x_1^{2n-3} & x_2^{2n-3} & \cdots & x_n^{2n-3}
\end{bmatrix}
}_{\text{Matriz de Vandermonde traspuesta}}
\begin{bmatrix}
w_1 \\
w_2 \\
\vdots \\
w_n
\end{bmatrix}
=
\begin{bmatrix}
2 \\
0 \\
\vdots \\
\int_{-1}^1 x^{2n-3} dx
\end{bmatrix}.
\]

\noindent La primera ecuación del sistema:
\[
\sum_{i=1}^n w_i = 2,
\]
es una condición necesaria para garantizar la exactitud en funciones constantes. Los pesos \( w_1, \ldots, w_n \) se calculan resolviendo este sistema, asegurando que la suma siempre sea 2 independientemente de \( n \). Por ejemplo, para \( n=3 \):

\[
\begin{cases}
w_1 + w_2 + w_3 = 2, \\
w_1(-1) + w_2 x_2 + w_3(1) = 0, \\
w_1(1) + w_2 x_2^2 + w_3(1) = \frac{2}{3}.
\end{cases}
\]

\noindent La solución \( w_1 = w_3 = \frac{1}{3} \), \( w_2 = \frac{4}{3} \), y \( x_2 = 0 \), cumple \( \sum w_i = 2 \), validando la propiedad general.

\begin{lstlisting}[frame=single, style=Matlab-Pyglike]
>> sum(w8)

ans =

    2.0000
\end{lstlisting}
\begin{lstlisting}[frame=single, style=Matlab-Pyglike]
>> sum(w9)

ans =

    2.0000
\end{lstlisting}
\end{itemize}

\subsection{Gauss-Lobatto para 8 y 9 nodos}

\begin{itemize}
\item \textbf{8 nodos}
\begin{lstlisting}[frame=single, style=Matlab-Pyglike]
>> [t8,w8] = GaussLobatto(8)

t8 =

    1.0000    0.8717    0.5917    0.2093   -0.2093   -0.5917   -0.8717   -1.0000


w8 =

    0.0357    0.2107    0.3411    0.4125    0.4125    0.3411    0.2107    0.0357
\end{lstlisting}

\item \textbf{9 nodos}
\begin{lstlisting}[frame=single, style=Matlab-Pyglike]
>> [t9,w9] = GaussLobatto(9)

t9 =

    1.0000    0.8998    0.6772    0.3631         0   -0.3631   -0.6772   -0.8998   -1.0000


w9 =

    0.0278    0.1655    0.2745    0.3464    0.3715    0.3464    0.2745    0.1655    0.0278
\end{lstlisting}

\end{itemize}
Ambos resultados tienen sentido pues
\begin{itemize}
\item Los vectores de nodos, \textit{t8} y \textit{t9}, son simétricos respecto del 0. %completar
\item Los vectores de pesos, \textit{w8} y \textit{w9}, son simétricos respecto del 0 y suman 2. %completar

\begin{lstlisting}[frame=single, style=Matlab-Pyglike]
>> sum(w8)

ans =

    2.0000
\end{lstlisting}
\begin{lstlisting}[frame=single, style=Matlab-Pyglike]
>> sum(w9)

ans =

    2.0000
\end{lstlisting}
\end{itemize}
\section{Implementación del programa de integración mediante Gauss-Lobatto}

\subsection{Gauss-Lobatto con extremos -1 y 1}
Queremos obtener un programa que tenga como parámetros de entrada la función \textit{f} a integrar, y el número de nodos \textit{n}, y de salida, el valor de la integral \textit{I}.

Primero, obtenemos el vector de nodos \textit{ti} y el de pesos \textit{wi} con el programa \textit{GaussLobatto.m} construido anteriormente. Después, con un bucle \textit{for} guardamos en el vector \textit{fi} los valores de la función evaluada en los nodos. Finalmente, asignamos a el vector \textit{s} el producto, componente a componente, del vector \textit{fi} y \textit{wi} y, usamos el comando \textit{sum} para realizar el sumatorio, cuyo valor guardamos en la variable \textit{I}.
\begin{lstlisting}[frame=single, numbers=left, style=Matlab-Pyglike]
function I = integraGaussLobatto(f,n)
%I = integraGaussLobatto(f,n)
[ti,wi]=GaussLobatto(n);
fi=[];
for i=1:n
    fi=[fi f(ti(i))];
end
s=fi.*wi;
I=sum(s);
end
\end{lstlisting}

Este código funciona para integrales no impropias con extremos de integración $-1$ y 1.

\subsection{Gauss-Lobatto con extremos a y b}
Este programa tiene los mismos parámetros de entrada que el anterior, además de los extremos de integración \textit{a} y \textit{b}, y el mismo parámetro de salida.

La mayor parte del procedimiento es igual al apartado anterior, solo hay que realizar el cambio de variable
\[x=\frac{b-a}{2}t-\frac{b+a}{2},\quad dx=\frac{b-a}{2}dt,\]

por lo que podemos aproximar

\[\displaystyle I=\int_{a}^{b} \! f(x)  \,dx = \frac{b-a}{2} \sum_{i=1}^{n}w_i f \left(\frac{b-a}{2}t_i + \frac{b+a}{2} \right)\]
\\
\begin{lstlisting}[frame=single, numbers=left, style=Matlab-Pyglike]
function I = integraGaussLobattoAB(a,b,f,n)
%I = integraGaussLobattoAB(f,n)
[ti,wi]=GaussLobatto(n);
ba=ones(1,n)*(b+a)/2;
ti=(ti*(b-a)/2)+ba;
fi=[];
for i=1:n
    fi=[fi f(ti(i))];
end
s=fi.*wi;
I=sum(s)*(b-a)/2;
end
\end{lstlisting}

\subsection{Gauss-Lobatto con extremos a y b para integrales impropias}

\subsubsection{Integral impropia como suma infinita de integrales propias}
Con estos dos programas, buscamos obtener la solución de integrales impropias mediante el método Gauss-Lobatto. Para ello,  podemos considerar una sucesión decreciente convergente a cero y aproximar el valor de la integral como suma
 finita de integrales propias.
 
A continuación, se probará que este método para sumas infinitas en efecto converge al valor de la integral impropia original:

Consideremos una sucesión $\{s_n\}_{n=1}^n$: $a+s_1<b$ $\wedge$ \stackrel{*} $s_k > s_{k-1} \forall k$ \wedge $\lim_{n \rightarrow \infty} s_n = 0$
y la siguiente suma finita de integrales propias:
$$
\int_{a + s_n}^b f(x) \, dx + \int_{a + s_{n-1}}^{a + s_n} f(x) \, dx + \cdots + \int_{a + s_1}^{a + s_2} f(x) \, dx \stackrel{*}= \int_a^b f(x) \, dx \quad $$
$$\Rightarrow \int_{a + s_1}^b f(x) \, dx + \int_{a + s_2}^{a + s_1} f(x) \, dx + \cdots = \lim_{n \rightarrow \infty} \int_{a + s_n}^b f(x) \, dx = \int_{a + 0}^b f(x) \, dx = \int_a^b f(x) \, dx$$

\subsubsection{Implementación del programa}

Haciendo uso de la técnica numérica del último apartado, queremos hacer una función para la resolución de una integral impropia con singularidad en el extremo inferior.

Se considera como sucesión decreciente $\{2^{-i}\}_{i=j}^{j+k-1}$ donde la k es un valor introducido por el usuario para optimizar la precisión de la solución (reflejada en la cantidad de sumas de integrales), y la j es un valor a determinar. Se toma la sucesion desde $j\geq 1$, dado a la posibilidad de que $a+1/2>b$ que estropearía la aproximación. Sin embargo, dado que se parte de una sucesión decreciente convergente a 0, se puede asegurar que $\exists j:a+2^{-j}<b$. Desde la línea \texttt{j=1;} hasta el \texttt{end} del bucle \texttt{while}, se busca la $j$ que cumpla la condición previamente mencionada. Desde el bucle \texttt{for} hasta el último \texttt{end} se implementa la suma finita de Integrales propias aproximadas por \texttt{integraGaussLobatto}.

 
\begin{lstlisting}[frame=single, numbers=left, style=Matlab-Pyglike]
function I = integraGaussLobattoImpropio(a,b,f,n,k)
%I = IntegraGaussLobattoImpropio(a,b,f,n,k)
I=0;
j=1;
a_end=a+1/(2^j);
while a_end>b
    j=j+1;
    a_end=a+1/(2^j);
end
for i=j:j+k-1
    ai=1/(2^i);
    I=I+integraGaussLobattoAB(a+ai/2,a+ai,f,n);
end
I=I+integraGaussLobattoAB(a_end,b,f,n);
end
\end{lstlisting}

\section{Aproximación mediante Gauss-Lobatto con 9 nodos de algunas integrales}
Para comprobar el resultado de las siguientes integrales usaremos el comando \textit{quadgk}

\begin{itemize}
    \item $\displaystyle I_1=\int_{0}^{\pi/3} \! \frac{1}{1-\sin(x)}  \,dx$
    \\
\begin{lstlisting}[frame=single, style=Matlab-Pyglike]
>> f=@(x) 1./(1-sin(x)); a=0; b=pi/3; n=9;
>> I1 = integraGaussLobattoAB(a,b,f,n)

I1 =

   2.732050857019763

>> I = quadgk(f,a,b)

I =

   2.732050807568876

>> Error = abs(I-I1)

Error =

     4.945088694086053e-08
\end{lstlisting}

El error absoluto entre ambos valores es de \(4.945088694086053 \times 10^{-8} \), lo que indica una excelente precisión. Esto se debe a que la función \( f(x) = \frac{1}{1-\sin(x)} \) es suave y tiene un buen comportamiento en el intervalo \( [0, \pi/3] \), permitiendo que la cuadratura de Gauss-Lobatto proporcione una aproximación muy cercana al valor real.

    \item $\displaystyle I_2=\int_{0}^{\pi} \! \frac{\sin(x)}{x^{3/2}}  \,dx$
    \\
\begin{lstlisting}[frame=single, style=Matlab-Pyglike]
>> f=@(x) sin(x)./x.^(3/2); a=0; b=pi; n=9; k=100;
>> I2 = integraGaussLobattoImpropio(a,b,f,n,k)

I2 =

   2.651469905786794

>> I = quadgk (f ,a , b )

I =

   2.651469253040811

>> Error = abs(I - I2)

Error =

     6.527459821548121e-07
\end{lstlisting}

Para \( I_2 \), se utilizó la función \texttt{integraGaussLobattoImpropio}  debido a la singularidad de la función \( f(x) = \frac{\sin(x)}{x^{3/2}} \) en \( x = 0 \).  El error absoluto es de \( 6.527459821548121 \times 10^{-7} \), lo que refleja una buena aproximación, aunque el error es mayor que en \( I_1 \). Esto se debe a la singularidad en \( x = 0 \), que introduce dificultades numéricas.

    \item $\displaystyle I_3=\int_{0}^{1} \! \frac{e^x}{x^2}  \,dx$
\\
\begin{lstlisting}[frame=single, style=Matlab-Pyglike]
>> f=@(x) exp(x)./x.^2; a=0; b=1; n=9; k=1000;
>> I3 = integraGaussLobattoImpropio(a,b,f,n,k)

I3 =

   Inf

>> I = quadgk (f ,a , b )
Warning: Infinite or Not-a-Number value encountered. 
> In quadgk>vadapt (line 281)
In quadgk (line 184) 

I =

   Inf
\end{lstlisting}

En el caso de \( I_3 \), tanto la cuadratura de Gauss-Lobatto como \textit{quadgk} devuelven un valor de \( \infty \). Esto indica que la integral diverge, lo cual es esperado debido a la singularidad de la función \( f(x) = \frac{e^x}{x^2} \) en \( x = 0 \). La función crece de manera extremadamente rápida cerca de \( x = 0 \), y la integral no converge en el intervalo \( [0, 1] \). Este resultado confirma que \( I_3 \) es una integral impropia que no tiene un valor finito.
    
\end{itemize}

\section{Comparación de resultados con Gauss-Legendre}
\begin{itemize}

    \item $\displaystyle I_1=\int_{0}^{\pi/3} \! \frac{1}{1-\sin(x)}  \,dx$
    \\
    \begin{lstlisting}[frame=single, style=Matlab-Pyglike]
>> f=@(x) 1./(1-sin(x)); a=0+1e-3; b=pi/3-1e-3; n=9;
>> I1Lob= integraGaussLobattoAB(a,b,f,n)

I1Lob =

   2.723600154858061

>> I1Leg= gaussLegendre(f,a,b,n)

I1Leg =

   2.723600103938956

>> Diferencia = abs(I1Lob - I1Leg)

Diferencia =

     5.091910448840054e-08

>> I = quadgk ( f ,a , b )

I =

   2.723600107261762

>> Error1Lob = abs(I - I1Lob)

Error1Lob =

     4.759629890216388e-08

>> Error1Leg= abs(I - I1Leg)

Error1Leg =

     3.322805586236655e-09

    \end{lstlisting}

En el cálculo de \( I_1 \), tanto Gauss-Legendre como Gauss-Lobatto ) con 9 nodos en el intervalo ajustado \( [0 + 10^{-3}, \pi/3 - 10^{-3}] \) ofrecen resultados precisos. Gauss-Legendre arroja un valor de \( 2.723600103938956 \) con un error absoluto de \( 3.322805586236655 \times 10^{-9} \), mientras que Gauss-Lobatto da \( 2.723600154858061 \) con un error de \( 4.759629890216388 \times 10^{-8} \), ambos respecto al valor de referencia de \textit{quadgk} (\( 2.723600107261762 \)). La diferencia entre ambos métodos es de \( 5.091910448840054 \times 10^{-8} \). Gauss-Legendre es más preciso en este caso, ya que la función \( f(x) = \frac{1}{1-\sin(x)} \) es suave.
\\
    \item $\displaystyle I_2=\int_{0}^{\pi} \! \frac{\sin(x)}{x^{3/2}}  \,dx$
    \\
    \begin{itemize}
    
    \item Con \textit{integraGaussLobattoAB} en el intervalo $[0+10^{-3},\pi-10^{-3}]$

\begin{lstlisting}[frame=single, style=Matlab-Pyglike]
>> f=@(x) sin(x)./x.^(3/2); a=0+1e-3; b=pi-1e-3; n=9;
>> I2Leg=gaussLegendre(f, a, b, n)

I2Leg =

   2.480811316956006

>> I2Lob=integraGaussLobattoAB(a,b,f,n)

I2Lob =

   3.564891079137420

>> I=quadgk(f,a,b)

I =

   2.588223612123681

>> Error2Leg=abs(I - I2Leg)

Error2Leg =

   0.107412295167674

>> Error2Lob=abs(I - I2Lob)

Error2Lob =

   0.976667467013740
   
>> Diferencia = abs(I2Lob - I2Leg)

Diferencia =

   1.084079762181414
\end{lstlisting}

En este caso, para \( I_2 \), se comparan Gauss-Legendre y Gauss-Lobatto  en el intervalo ajustado \( [10^{-3}, \pi - 10^{-3}] \). Gauss-Legendre da un valor de \( 2.480811316956006 \) con un error absoluto de \( 0.107412295167674 \), mientras que Gauss-Lobatto arroja \( 3.564891079137420 \) con un error de \( 0.976667467013740 \), ambos respecto al valor de referencia de \textit{quadgk} (\( 2.588223612123681 )\). La diferencia entre ambos métodos es de \( 1.084079762181414 \). Ambos métodos tienen errores significativos debido a la singularidad de \( f(x) = \frac{\sin(x)}{x^{3/2}} \) en \( x = 0 \), que afecta la precisión incluso en un intervalo ajustado. Gauss-Legendre tiene un error menor que Gauss-Lobatto, pero ambos son imprecisos en este contexto.
    
    \item Con \textit{integraGaussLobattoImpropio}
\begin{lstlisting}[frame=single, style=Matlab-Pyglike]
>> f=@(x) sin(x)./x.^(3/2); a=0+1e-3; b=pi-1e-3; n=9;
>> I2Leg=gaussLegendre(f, a,b, n)

I2Leg =

   2.480811316956006

>> I2Lob = integraGaussLobattoImpropio (0 ,pi ,f ,n , 100)

I2Lob =

2.651469905786794

>> I = quadgk(f ,a , b )

I =

   2.588223612123681

>> Error2Leg=abs(I-I2Leg)

Error2Leg =

   0.107412295167674

>> Error2Lob = abs( I - I2Lob )

Error2Lob =

6.527459821548121e-07
\end{lstlisting}
\\
Al usar \texttt{integraGaussLobattoImpropio} para \( I_2 \) con \( k = 100 \) en el intervalo original \( [0, \pi] \), Gauss-Lobatto mejora significativamente, dando un valor de \( 2.651469905786794 \) con un error de \( 6.527459821548121 \times 10^{-7} \), mucho menor que el error de Gauss-Legendre (\( 0.107412295167674 \)) en el intervalo ajustado \( [10^{-3}, \pi - 10^{-3}] \). Esto demuestra que \texttt{integraGaussLobattoImpropio} maneja mejor la singularidad en \( x = 0 \), ya que está diseñado para integrales impropias, mientras que Gauss-Legendre y \texttt{integraGaussLobattoAB} no están optimizados para este tipo de problemas.
\end{itemize}
\\

\item $\displaystyle I_3=\int_{0}^{1} \! \frac{e^x}{x^2}  \,dx$
\\
\begin{itemize}
    \item Con \textit{integraGaussLobattoAB} en el intervalo $[0+10^{-3},1-10^{-3}]$
    \\
    \begin{lstlisting}[frame=single, style=Matlab-Pyglike]
>> f=@(x) exp(x)./x.^2; a=0+1e-3; b=1-1e-3; n=9;
>> I3Leg=gaussLegendre(f,a,b,n)

I3Leg =

     1.676112081728000e+02

>> I3Lob=integraGaussLobattoAB(a,b,f,n)

I3Lob =

     1.391952022433983e+04

>> Diferencia= abs(I3Lob - I3Leg)

Diferencia =

     1.375190901616703e+04

>> I = quadgk(f,a,b)

I =

     1.006504155876315e+03

>> Error3Leg=abs(I-I3Leg)

Error3Leg =

     8.388929477035150e+02

>> Error3Lob=abs(I-I3Lob)

Error3Lob =

     1.291301606846351e+04
    \end{lstlisting}
    \\
Para \( I_3 \), en el intervalo ajustado \( [10^{-3}, 1 - 10^{-3}] \), se obtienen resultados muy imprecisos. Gauss-Legendre da \( 167.6112081728 \) con un error de \( 838.892947703515 \), mientras que Gauss-Lobatto arroja \( 13919.52022433983 \) con un error de \( 12913.01606846351 \), ambos respecto al valor de \textit{quadgk} (\( 1006.504155876315 \)). La diferencia entre ambos métodos es de \( 13751.90901616703 \). La función \( f(x) = \frac{e^x}{x^2} \) tiene una singularidad fuerte en \( x = 0 \), y aunque el intervalo está ajustado, ambos métodos fallan en capturar el comportamiento de la integral, siendo Gauss-Lobatto particularmente impreciso.    
    \item Con \textit{integraGaussLobattoImpropio}
    \\
\begin{lstlisting}[frame=single, style=Matlab-Pyglike]
>> f=@(x) exp(x)./x.^2; a=0+1e-3; b=1-1e-3; n=9;
>> I3Leg=gaussLegendre(f, a,b, n)

I3Leg =

     1.676112081728000e+02

>> I3 = integraGaussLobattoImpropio(0,1,f,n,100)

I3 =

   Inf

>> I = quadgk(f ,a , b )

I =

     1.006504155876315e+03

>> Error3Leg=abs(I-I3Leg)

Error3Leg =

     8.388929477035150e+02
\end{lstlisting}
    \\
Al emplear \texttt{integraGaussLobattoImpropio} con \( k = 100 \) en el intervalo original \( [0, 1] \), Gauss-Lobatto identifica correctamente que la integral diverge (\( \text{Inf} \)), lo cual es coherente, ya que \( f(x) = \frac{e^x}{x^2} \) crece rápidamente cerca de \( x = 0 \). En contraste, Gauss-Legendre en el intervalo ajustado subestima la integral (\( 167.6112081728 \), error de \( 838.892947703515 \)), mostrando su limitación para integrales con singularidades fuertes, mientras que \\ \texttt{integraGaussLobattoImpropio} captura adecuadamente la naturaleza divergente de \( I_3 \).

\end{itemize}
\end{itemize}
\section{Bibliografía}
\begin{itemize}
    \item Análisis Numérico (Burden, Richard L.)
    \item Análisis Numérico con Aplicaciones (Gerald, Curtis F.)
    \item Problemas Resueltos de Métodos Numéricos (Torregrosa Sánchez, Juan Ramón $\vert$ Hueso Pagoaga, José Luis $\vert$ Cordero Barbero, Alicia $\vert$ Martínez Molada, Eulalia)
\end{itemize}
\end{document}
