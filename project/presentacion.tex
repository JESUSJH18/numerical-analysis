\documentclass{beamer}

%% Configuración de la presentación
\mode<presentation> {
  %%% Selección de estilo
  % The Beamer class comes with a number of default slide themes
  % which change the colors and layouts of slides. Below this is a list
  % of all the themes, uncomment each in turn to see what they look like.

  \usetheme{Warsaw}


  %% Configuración del pie de línea
  %\setbeamertemplate{footline} % To remove the footer line in all slides uncomment this line
  %\setbeamertemplate{footline}[page number] % To replace the footer line in all slides with a simple slide count uncomment this line
  %\setbeamertemplate{navigation symbols}{} % To remove the navigation symbols from the bottom of all slides uncomment this line
}

%% Fuentes de tamaño arbitrario
\usepackage{lmodern}

%% Gráficos
\usepackage{graphicx} % Allows including images
\usepackage{booktabs} % Allows the use of \toprule, \midrule and \bottomrule in tables

%%% Castellano.
% noquoting: Permite uso de comillas no españolas.
% lcroman: Permite la enumeración con numerales romanos en minúscula.
% fontenc: Usa la fuente completa para que pueda copiarse correctamente del pdf.
\usepackage[spanish,es-noquoting,es-lcroman]{babel}
\usepackage[utf8]{inputenc}
\usepackage[T1]{fontenc}
\selectlanguage{spanish}

%----------------------------------------------------------------------------------------
%	TÍTULO
%----------------------------------------------------------------------------------------


\title{Integración numérica de Gauss-Lobatto} % The short title appears at the bottom of every slide, the full title is only on the title page

\author[Jesús de Juan, Jürgen León, Pablo Navarro, María C. Sánchez]{Jesús de Juan López \\
        Jürgen León Martín \\
        Pablo Navarro Rubio \\
        María Cruz Sánchez Montoya} % Your name
\institute[UPV] % Your institution as it will appear on the bottom of every slide, may be shorthand to save space
{ Trabajo I4, Grupo 2 \\ % Your institution for the title page
}
\date{Universidad Politécnica de Valencia} % Date, can be changed to a custom date


\begin{document}

%% Diapositiva de título.
\begin{frame}
\titlepage % Print the title page as the first slide
\end{frame}

%% Diapositiva de contenidos.
% Throughout your presentation, if you choose to use \section{} and \subsection{} commands, 
% these will automatically be printed on this slide as an overview of your presentation
\begin{frame}
  \frametitle{Índice} % Table of contents slide, comment this block out to remove it
  \tableofcontents
\end{frame}



%----------------------------------------------------------------------------------------
%	PRESENTACIÓN
%----------------------------------------------------------------------------------------

%------------------------------------------------
\section{Introducción} % Sections can be created in order to organize your presentation into discrete blocks, all sections and subsections are automatically printed in the table of contents as an overview of the talk
%------------------------------------------------

%\subsection{Subsection Example} % A subsection can be created just before a set of slides with a common theme to further break down your presentation into chunks

\begin{frame}
\frametitle{Introducción}
Fórmula de cuadratura de \textcolor{blue}{Gauss-Lobatto} para \textcolor{blue}{n} nodos:
\[
\int_{-1}^{1} \! f(t)  \,dt = w_1 f(-1) + w_n f(1) + \sum_{i=1}^{n-1}w_i f(t_i),
\]
\\~\\

donde $t_i$, $i=2, 3,...,n-1$, son las raíces del polinomio $p'_{n-1}(t)$, siendo $p_{n-1}(t)$ el polinomio de Legendre de grado $n-1$. Los \textcolor{blue}{pesos} son $w_1 = w_n = \tfrac{2}{n(n-1)}$ y
\[
w_i=\frac{2}{n(n-1)(p_{n-1}(t_i))^2},\qquad i=2,3,...,n-1.
\].
\end{frame}

%------------------------------------------------
\section{Construcción del programa de Gauss-Lobatto}
\begin{frame}
\frametitle{Construcción del programa de Gauss-Lobatto}

\begin{table}
    \centering
    \begin{tabular}{cc}
    \toprule
   \textbf{ Parámetros de entrada} & \textbf{Parámetros de salida} \\
   \midrule
    Número de nodos \textcolor{blue}{n} & Vector de nodos \textcolor{blue}{ti}\\
         & Vector de pesos \textcolor{blue}{wi}\\
    \bottomrule
    \end{tabular}

    \label{tab:my_label}
\end{table}
\begin{enumerate}
\item Evaluar los \textcolor{blue}{posibles casos} según los nodos pedidos.
\item El vector de nodos intermedios \textcolor{blue}{ti}: Polinomio de Legendre de grado $n-1$, su derivada y las raíces de esta.
\item El vector de pesos \textcolor{blue}{wi}.
\item El vector de nodos completo \textcolor{blue}{ti} y programa final.
\end{enumerate}
\end{frame}

%------------------------------------------------
\subsection{Posibles casos según los nodos pedidos}
\begin{frame}
\frametitle{Posibles casos según los nodos pedidos}
\begin{block}{$n<2$}
No es posible calcular  la cuadratura por:
\begin{itemize}
    \item $n=1$:
    \begin{itemize}
        \item No hay suficientes nodos para incluir los extremos de la cuadratura $−1$ y $1$.
        \item Con un único nodo no captaría ninguna curvatura, luego no hay forma de aproximar con precisión.
    \end{itemize}
    \item $n<1$: 
    \begin{itemize}
        \item No tiene sentido calcular una cuadratura sin nodos, o con nodos negativos.
    \end{itemize}
\end{itemize}
\end{block}

\begin{block}{$n=2$}
El vector de nodos $t_i=[-1\quad 1]$, y solo hay que calcular $w_i=[w_1\quad w_n]$.
\end{block}
\end{frame}

\begin{frame}
\frametitle{Posibles casos según los nodos pedidos}
\begin{block}{$n>2$}
Suspendisse tincidunt sagittis gravida. Curabitur condimentum, enim sed venenatis rutrum, ipsum neque consectetur orci, sed blandit justo nisi ac lacus.
\end{block}
\end{frame}

%------------------------------------------------

\begin{frame}
\frametitle{Multiple Columns}
\begin{columns}[c] % The "c" option specifies centered vertical alignment while the "t" option is used for top vertical alignment

\column{.45\textwidth} % Left column and width
\textbf{Heading}
\begin{enumerate}
\item Statement
\item Explanation
\item Example
\end{enumerate}

\column{.5\textwidth} % Right column and width
Lorem ipsum dolor sit amet, consectetur adipiscing elit. Integer lectus nisl, ultricies in feugiat rutrum, porttitor sit amet augue. Aliquam ut tortor mauris. Sed volutpat ante purus, quis accumsan dolor.

\end{columns}
\end{frame}

%------------------------------------------------
\section{Gauss-Lobatto para 8 y 9 nodos}
%------------------------------------------------

\begin{frame}
\frametitle{Gauss-Lobatto para 8 y 9 nodos}

\end{frame}

%------------------------------------------------

\begin{frame}
\frametitle{Theorem}
\begin{theorem}[Mass--energy equivalence]
$E = mc^2$
\end{theorem}
\end{frame}

%------------------------------------------------

\begin{frame}[fragile] % Need to use the fragile option when verbatim is used in the slide
\frametitle{Verbatim}
\begin{example}[Theorem Slide Code]
\begin{verbatim}
\begin{frame}
\frametitle{Theorem}
\begin{theorem}[Mass--energy equivalence]
$E = mc^2$
\end{theorem}
\end{frame}\end{verbatim}
\end{example}
\end{frame}

%------------------------------------------------

\begin{frame}
\frametitle{Figure}
Uncomment the code on this slide to include your own image from the same directory as the template .TeX file.
%\begin{figure}
%\includegraphics[width=0.8\linewidth]{test}
%\end{figure}
\end{frame}
%------------------------------------------------

\begin{frame}[fragile] % Need to use the fragile option when verbatim is used in the slide
\frametitle{Verbatim}
\begin{example}[Gauss-Lobatto con extremos -1 y 1]
\begin{verbatim}
function I = integraGaussLobatto(f,n)
%I = integraGaussLobatto(f,n)
[ti,wi]=GaussLobatto(n);
fi=[];
for i=1:n
    fi=[fi f(ti(i))];
end
s=fi.*wi;
I=sum(s);
end\end{verbatim}
\end{example}
\end{frame}

%------------------------------------------------
%------------------------------------------------

\begin{frame}[fragile] % Need to use the fragile option when verbatim is used in the slide
\frametitle{Verbatim}
\begin{example}[Gauss-Lobatto con extremos a y b]
\begin{verbatim}
function I = integraGaussLobattoAB(a,b,f,n)
%I = integraGaussLobattoAB(f,n)
[ti,wi]=GaussLobatto(n);
ba=ones(1,n)*(b+a)/2;
ti=(ti*(b-a)/2)+ba;
fi=[];
for i=1:n
    fi=[fi f(ti(i))];
end
s=fi.*wi;
I=sum(s)*(b-a)/2;
end\end{verbatim}
\end{example}
\end{frame}

%------------------------------------------------
%------------------------------------------------

\begin{frame}[fragile] % Need to use the fragile option when verbatim is used in the slide
\frametitle{Citation}
An example of the \verb|\cite| command to cite within the presentation:\\~

This statement requires citation \cite{p1}.
\end{frame}

%------------------------------------------------

%% Bibliografía
\begin{frame}
\frametitle{Referencias}
\footnotesize{
  \begin{thebibliography}{99} % Beamer does not support BibTeX so references must be inserted manually as below
    \bibitem[Smith, 2012]{p1} John Smith (2012)
      \newblock Title of the publication
      \newblock \emph{Journal Name} 12(3), 45 -- 678.
  \end{thebibliography}
}
\end{frame}

%------------------------------------------------

\begin{frame}
\Huge{\centerline{Fin.}}
\end{frame}

%----------------------------------------------------------------------------------------

\end{document} 