\documentclass{beamer}






%% Configuración de la presentación
\mode<presentation> {
  %%% Selección de estilo
  % The Beamer class comes with a number of default slide themes
  % which change the colors and layouts of slides. Below this is a list
  % of all the themes, uncomment each in turn to see what they look like.

  \usetheme{Warsaw}


  %% Configuración del pie de línea
  %\setbeamertemplate{footline} % To remove the footer line in all slides uncomment this line
  %\setbeamertemplate{footline}[page number] % To replace the footer line in all slides with a simple slide count uncomment this line
  %\setbeamertemplate{navigation symbols}{} % To remove the navigation symbols from the bottom of all slides uncomment this line
}

%% Fuentes de tamaño arbitrario
\usepackage{lmodern}

%% Gráficos

\usepackage{colortbl}
\usepackage[table]{xcolor}
\usepackage{graphicx} % Allows including images
\usepackage{booktabs} % Allows the use of \toprule, \midrule and \bottomrule in tables

%%% Castellano.
% noquoting: Permite uso de comillas no españolas.
% lcroman: Permite la enumeración con numerales romanos en minúscula.
% fontenc: Usa la fuente completa para que pueda copiarse correctamente del pdf.
\usepackage[spanish,es-noquoting,es-lcroman]{babel}
\usepackage[utf8]{inputenc}
\usepackage[T1]{fontenc}
\selectlanguage{spanish}


%----------------------------------------------------------------------------------------
%	TÍTULO
%----------------------------------------------------------------------------------------


\title{Integración numérica de Gauss-Lobatto} % The short title appears at the bottom of every slide, the full title is only on the title page

\author[Jesús de Juan, Jürgen León, Pablo Navarro, María C. Sánchez]{Jesús de Juan López \\
        Jürgen León Martín \\
        Pablo Navarro Rubio \\
        María Cruz Sánchez Montoya} % Your name
\institute[UPV] % Your institution as it will appear on the bottom of every slide, may be shorthand to save space
{ Trabajo I4, Grupo 2 \\ % Your institution for the title page
}
\date{Universidad Politécnica de Valencia} % Date, can be changed to a custom date


\begin{document}

%% Diapositiva de título.
\begin{frame}
\titlepage % Print the title page as the first slide
\end{frame}

%% Diapositiva de contenidos.
% Throughout your presentation, if you choose to use \section{} and \subsection{} commands, 
% these will automatically be printed on this slide as an overview of your presentation
\begin{frame}
  \frametitle{Índice} % Table of contents slide, comment this block out to remove it
  \tableofcontents
\end{frame}



%----------------------------------------------------------------------------------------
%	PRESENTACIÓN
%----------------------------------------------------------------------------------------

%------------------------------------------------
\section{Introducción} % Sections can be created in order to organize your presentation into discrete blocks, all sections and subsections are automatically printed in the table of contents as an overview of the talk
%------------------------------------------------

%\subsection{Subsection Example} % A subsection can be created just before a set of slides with a common theme to further break down your presentation into chunks

\begin{frame}
\frametitle{Introducción}
Fórmula de cuadratura de \textcolor{blue}{Gauss-Lobatto} para \textcolor{blue}{n} nodos:
\[
\int_{-1}^{1} \! f(t)  \,dt = w_1 f(-1) + w_n f(1) + \sum_{i=1}^{n-1}w_i f(t_i),
\]
\\~\\

donde $t_i$, $i=2, 3,...,n-1$, son las raíces del polinomio $p'_{n-1}(t)$, siendo $p_{n-1}(t)$ el polinomio de Legendre de grado $n-1$. Los \textcolor{blue}{pesos} son $w_1 = w_n = \tfrac{2}{n(n-1)}$ y
\[
w_i=\frac{2}{n(n-1)(p_{n-1}(t_i))^2},\qquad i=2,3,...,n-1.
\].
\end{frame}

%------------------------------------------------
\section{Construcción del programa de Gauss-Lobatto}
\begin{frame}
\frametitle{Construcción del programa de Gauss-Lobatto}

\begin{table}
    \centering
    \begin{tabular}{cc}
    \toprule
   \textbf{ Parámetros de entrada} & \textbf{Parámetros de salida} \\
   \midrule
    Número de nodos \textcolor{blue}{n} & Vector de nodos \textcolor{blue}{ti}\\
         & Vector de pesos \textcolor{blue}{wi}\\
    \bottomrule
    \end{tabular}

    \label{tab:my_label}
\end{table}
\begin{enumerate}
\item Evaluar los \textcolor{blue}{posibles casos} según los nodos pedidos.
\item El vector de nodos intermedios \textcolor{blue}{ti}: Polinomio de Legendre de grado $n-1$, su derivada y las raíces de esta.
\item El vector de pesos \textcolor{blue}{wi}.
\item El vector de nodos completo \textcolor{blue}{ti} y programa final.
\end{enumerate}
\end{frame}

%------------------------------------------------
\subsection{Posibles casos según los nodos pedidos}
\begin{frame}
\frametitle{Posibles casos según los nodos pedidos}
\begin{block}{$n<2$}
No es posible calcular  la cuadratura por:
\begin{itemize}
    \item $n=1$:
    \begin{itemize}
        \item No hay suficientes nodos para incluir los extremos de la cuadratura $−1$ y $1$.
        \item Con un único nodo no captaría ninguna curvatura, luego no hay forma de aproximar con precisión.
    \end{itemize}
    \item $n<1$: 
    \begin{itemize}
        \item No tiene sentido calcular una cuadratura sin nodos, o con nodos negativos.
    \end{itemize}
\end{itemize}
Se mostrará el mensaje \textit{"Se necesitan al menos 2 nodos para la cuadratura"}.
\end{block}

\end{frame}

\begin{frame}
\frametitle{Posibles casos según los nodos pedidos}

\begin{block}{$n=2$}
El vector de nodos $ti=[-1\quad 1]$, y solo hay que calcular el vector de pesos $wi=[w_1\quad w_n]=\left[\tfrac{2}{n(n-1)}\quad \tfrac{2}{n(n-1)}\right]$.
\end{block}

\begin{block}{$n>2$} 
El vector de nodos $ti=[-1\quad t_i \quad 1]$, donde $t_i$, $i=2, 3,...,n-1$, son las raíces del polinomio $p'_{n-1}(t)$, siendo $p_{n-1}(t)$ el polinomio de Legendre de grado $n-1$.

El vector de pesos 

$wi=[w_1 \quad w_i \quad w_n]= \left[\tfrac{2}{n(n-1)} \quad \frac{2}{n(n-1)(p_{n-1}(t_i))^2} \quad \tfrac{2}{n(n-1)}\right]$
.
\end{block}
\end{frame}

%------------------------------------------------
\subsection{El vector de nodos intermedios $ti$}
%------------------------------------------------
\begin{frame}[fragile]
\frametitle{El vector de nodos intermedios $ti$}

\textbf{Polinomio de Legendre de orden $n-1$}
\begin{itemize}
    \item Código auxiliar $LegendreT.m$, que devuelve los coeficientes truncados.
    \item Lo guardamos en el vector $cl$. \\
    \begin{exampleblock}{Código}
    \begin{verbatim}
    cl=LegendreT(n-1)
    \end{verbatim}
    \end{exampleblock}
     
   
    
\end{itemize}



\end{frame}
\begin{frame}[fragile]
\frametitle{El vector de nodos intermedios $ti$}



\textbf{Derivada del polinomio de Legendre de orden $n-1$.}
\\ Aplicamos la regla de derivación de polinomios:
\[
\frac{d}{dx}(a_k x^k)=k a_k x^{k-1}
\]
Los coeficientes de la derivada los guardaremos en el vector \textit{ci}.
La derivada del polinomio de grado $n-1$ tiene grado $n-2$, eliminamos el último término correspondiente al grado 0, cuya derivada es 0. \\ 
\begin{exampleblock}{Código}
\begin{verbatim}
ci=((n-1):-1:0).*cl; 
ci=ci(:,(1:n-1));
\end{verbatim}
\end{exampleblock}


\end{frame}
\begin{frame}[fragile]
\frametitle{El vector de nodos intermedios $ti$}

\textbf{Raíces de la derivada del polinomio de Legendre de orden $n-1$.}
Utilizando el comando \textit{root}, pero como nos da un vector columna (al que llamamos $ti$) lo tenemos que trasponer. Queremos trabajar con vectores fila y ordenamos de forma descendente.

\textit{\\Código:}
\begin{exampleblock}{Código}
\begin{verbatim}
ti=roots(ci);
ti=ti';
ti=sort(ti,"descend");
\end{verbatim}
\end{exampleblock}
\end{frame}

%------------------------------------------------
\subsection{El vector de pesos $wi$}
%------------------------------------------------
\begin{frame}
\frametitle{El vector de pesos $wi$}
Asignamos a la variable \textit{wn} el valor de los pesos de los extremos $wn=w_1=w_n$, e inicializamos el vector \textit{wi} vacío.

Con un bucle \textcolor{blue}{for} calculamos los pesos intermedios 
\[
w_i=w_n\frac{1}{(p_{n-1}(t_i))^2},\qquad i=2,3,...,n-1.
\]
que vamos añadiendo al vector \textit{wi}.

Agregamos los pesos de los extremos \textit{wn} al vector \textit{wi}, obteniendo el vector de pesos \textit{wi}.
\end{frame}

\begin{frame}[fragile]
\frametitle{El vector de pesos $wi$}
\begin{exampleblock}{Código}
\begin{verbatim}
for i=1:(n-2)
    pi=polyval(cl,ti(i));
    w=wn/(pi^2);
    wi=[wi w];
end
wi=[wn wi wn];
\end{verbatim}
\end{exampleblock}
\end{frame}

%------------------------------------------------
\subsection{El vector de nodos completo \textit{ti}}
%------------------------------------------------
\begin{frame}[fragile]
\frametitle{El vector de nodos completo \textit{ti}}
Solo falta añadir los nodos de los extremos $-1$ y 1 al vector \textit{ti}. Con esto, ya tenemos el vector de nodos \textit{ti} completo.

\begin{exampleblock}{Código}
\begin{verbatim}
ti=[1 ti -1];
\end{verbatim}
\end{exampleblock}
\end{frame}

%------------------------------------------------
\subsection{Programa final}
%------------------------------------------------
\begin{frame}[fragile]
\frametitle{Programa final}
\begin{exampleblock}{Gauss-Lobatto}
\begin{verbatim}
function [ti,wi] = GaussLobatto(n)
%[ti,wi] = GaussLobatto(n)
wi=[];
wn=2/(n*(n-1));
if n>2
    cl=LegendreT(n-1);
    ci=((n-1):-1:0).*cl;
    ci=ci(:,(1:n-1));
    ti=roots(ci);
    ti=ti';
    ti=sort(ti,"descend");
    for i=1:(n-2)
        pi=polyval(cl,ti(i));
\end{verbatim}
\end{exampleblock}
\end{frame}

\begin{frame}[fragile]
\frametitle{Programa final}
\begin{exampleblock}{}
\begin{verbatim}
        w=wn/(pi^2);
        wi=[wi w];
    end
    wi=[wn wi wn];
    ti=[1 ti -1];
else 
    if n==2
        wi=[wn wn];
        ti=[1 -1];
    else
    disp("Se necesitan al menos 2 nodos para la 
    cuadratura")
    end
end
end
\end{verbatim}
\end{exampleblock}
\end{frame}
%------------------------------------------------
\section{Gauss-Lobatto para 8 y 9 nodos}
%------------------------------------------------

\begin{frame}
\frametitle{Gauss-Lobatto para 8 y 9 nodos}
Para justificar los resultados obtenidos por el programa \texttt{GaussLobatto} nos basaremos en un par de propiedades de los pesos y los nodos que demostraremos a continuación:
\begin{itemize}
    \item Simetría de nodos respecto al 0
    \item La suma de todos los pesos es igual a 2 y son simétricos
\end{itemize}
\end{frame}

%------------------------------------------------
\subsection{Simetría de nodos y pesos}
%------------------------------------------------
\begin{frame}
\frametitle{Paridad e imparidad de polinomios de Legendre}
Queremos probar que el polinomio de Legendre $l_i$ es función par si $i=2k$ e impar si $i=2k+1$, $k\in\mathbb{N}\cup\{0\}$ por el método de inducción. Podemos ver que para los casos $i=0, 1, 2, 3$ se cumple:
\[l_0=1\quad;\quad l_1=x\quad;\quad l_2=x^2-\frac{1}{3}\quad;\quad l_3=x^3-\frac{3}{5}x \]
Supongamos que $l_i$ es par y $l_{i+1}$ impar y probemos para $l_{i+2}$ y $l_{i+3}$ :
\end{frame}
\begin{frame}
\frametitle{Paridad e imparidad de polinomios de Legendre}
a) Caso par:
\[l_{i+2}(x)=(x-b_{i+2})\cdot l_{i+1}(x)-c_{i+2}\cdot l_i(x)\quad, \quad\alpha_{i+1}=\int_{-1}^1l_{i+1}^2(x)dx>0\]
Por lo tanto \[b_{i+2}=\frac{1}{\alpha_{i+1}} \int_{-1}^1\underbrace{x\cdot l_{i+1}^2(x)}_{impar}dx=0\]
\[\Rightarrow l_{i+2}=\underbrace{x\cdot l_{i+1}(x)}_{par}-\underbrace{c_{i+2}\cdot l_i(x)}_{par}\Rightarrow l_{i+2}(x)\text{ es par.} \]
\end{frame}
\begin{frame}
\frametitle{Paridad e imparidad de polinomios de Legendre}
b) Caso impar:
\[l_{i+3}(x)=(x-b_{i+3})\cdot l_{i+2}(x)-c_{i+3}\cdot l_{i+1}(x)\quad,\quad\alpha_{i+2}=\int_{-1}^1l_{i+2}^2(x)dx>0\] 
Por lo tanto \[b_{i+3}=\frac{1}{\alpha_{i+2}}\int_{-1}^1\underbrace{x\cdot l_{i+2}^2(x)}_{impar}dx=0\]
\[\Rightarrow b_{i+3}=\underbrace{x\cdot l_{i+2}(x)}_{impar}-\underbrace{c_{i+3}\cdot l_{i+1}(x)}_{impar}\Rightarrow l_{i+3}(x)\text{ es impar.}\]
\end{frame}
%------------------------------------------------

\begin{frame} 
\frametitle{Simetría de raíces para funciones pares e impares}
\begin{block}{Caso par} 
Supongamos que existe una raíz $x_i:f(x_i)=0$. Al ser $f(x)$ una función par, se cumple que $f(-x_i)=f(x_i)=0$. Por lo tanto, se cumple que para cada raíz de la función par, su elemento opuesto también es raíz.
\end{block}
\begin{block}{Caso impar} 
Supongamos que existe una raíz $x_i:f(x_i)=0$. Al ser $f(x)$ impar, se cumple que $-f(-x_i)=f(x_i)=0\Rightarrow f(-x_i)=0$. Por lo tanto, se cumple que para cada raíz de la función par, su elemento opuesto también es raíz.
\end{block}
\end{frame}

%------------------------------------------------

\begin{frame}
\frametitle{Simetría nodos y pesos}
Utilizando el primer resultado, vemos que las derivadas de los polinomios de Legendre truncados también deben ser o pares o impares, dado que al derivar el polinomio se invierte la paridad. Por el segundo resultado, el conjunto de raíces de $\tilde l_i(x)'$ (que corresponden con los nodos de Gauss-Lobatto) debe ser simétrico entorno al 0. Dado que los pesos se calculan a partir de $(p_{n-1}(t_i))^2$, y teniendo en cuenta que $p_{n-1}^2(x)$ debe ser par y que los nodos son simétricos, entonces los pesos también lo deben ser. 
\end{frame}
%------------------------------------------------
\subsection{Suma de pesos}
%------------------------------------------------
\begin{frame}

\end{frame}
%------------------------------------------------
\section{Integracion mediante Gauss-Lobatto}
%------------------------------------------------
\subsection{Gauss-Lobatto con extremos -1 y 1}
\begin{frame}[fragile] % Need to use the fragile option when verbatim is used in the slide
\frametitle{Gauss-Lobatto con extremos -1 y 1}
\begin{exampleblock}{Gauss-Lobatto con extremos -1 y 1}
\begin{verbatim}
function I = integraGaussLobatto(f,n)
%I = integraGaussLobatto(f,n)
[ti,wi]=GaussLobatto(n);
fi=[];
for i=1:n
    fi=[fi f(ti(i))];
end
s=fi.*wi;
I=sum(s);
end
\end{verbatim}
\end{exampleblock}
\end{frame}

%------------------------------------------------

%------------------------------------------------


\subsection{Gauss-Lobatto con extremos a y b}
\begin{frame}
\frametitle{Gauss-Lobatto con extremos a y b}
El procedimiento es igual al apartado anterior, solo hay que realizar el \textcolor{blue}{cambio de variable}:
\[x=\frac{b-a}{2}t-\frac{b+a}{2},\quad dx=\frac{b-a}{2}dt,\]
\\~\\
por lo que podemos aproximar
\[\displaystyle I=\int_{a}^{b} \! f(x)  \,dx = \frac{b-a}{2} \sum_{i=1}^{n}w_i f \left(\frac{b-a}{2}t_i + \frac{b+a}{2} \right)\]
\end{frame}

%------------------------------------------------

\begin{frame}[fragile] % Need to use the fragile option when verbatim is used in the slide
\frametitle{Gauss-Lobatto con extremos a y b}
\begin{exampleblock}{Gauss-Lobatto con extremos a y b}
\begin{verbatim}
function I = integraGaussLobattoAB(a,b,f,n)
%I = integraGaussLobattoAB(f,n)
[ti,wi]=GaussLobatto(n);
ba=ones(1,n)*(b+a)/2;
ti=(ti*(b-a)/2)+ba;
fi=[];
for i=1:n
    fi=[fi f(ti(i))];
end
s=fi.*wi;
I=sum(s)*(b-a)/2;
end\end{verbatim}
\end{exampleblock}
\end{frame}

%------------------------------------------------
%------------------------------------------------


\subsection{Gauss-Lobatto para integrales impropias}
\begin{frame}
\frametitle{Gauss-Lobatto para integrales impropias}
Consideremos la sucesion decreciente convergente a cero $\{s_n\}_{n=1}^n$: $a+s_1<b$ $\wedge$ \stackrel{*} $s_k > s_{k-1} \forall k$ \wedge $\lim_{n \rightarrow \infty} s_n = 0$
y la siguiente suma finita de integrales propias para probar que este metodo para sumas infinitas en efecto converge al valor de
la integral impropia original
$$
\int_{a + s_n}^b f(x) \, dx + \int_{a + s_{n-1}}^{a + s_n} f(x) \, dx + \cdots + \int_{a + s_1}^{a + s_2} f(x) \, dx \stackrel{*}= \int_a^b f(x) \, dx \quad $$
$$\Rightarrow \int_{a + s_1}^b f(x) \, dx + \int_{a + s_2}^{a + s_1} f(x) \, dx + \cdots = \lim_{n \rightarrow \infty} \int_{a + s_n}^b f(x) \, dx $$$$ =\int_{a + 0}^b f(x) \, dx = \int_a^b f(x) \, dx$$
\\~\\
\end{frame}

%------------------------------------------------
%------------------------------------------------

\begin{frame}[fragile] % Need to use the fragile option when verbatim is used in the slide
\frametitle{Gauss-Lobatto para integrales impropias}
\begin{exampleblock}{Gauss-Lobatto para integrales impropias}
\begin{verbatim}
function I = integraGaussLobattoImpropio(a,b,f,n,k)
%I = IntegraGaussLobattoImpropio(a,b,f,n,k)
I=0;
j=1;
a_end=a+1/(2^j);
while a_end>b
    j=j+1;
    a_end=a+1/(2^j);
end
for i=j:j+k-1
    ai=1/(2^i);
    I=I+integraGaussLobattoAB(a+ai/2,a+ai,f,n);
end
I=I+integraGaussLobattoAB(a_end,b,f,n);
end\end{verbatim}
\end{exampleblock}
\end{frame}

%------------------------------------------------
\section{Aproximación mediante Gauss-Lobatto con 9 nodos}
%------------------------------------------------

\begin{frame}
\frametitle{Aproximación mediante Gauss-Lobatto con 9 nodos}

Para comprobar el resultado de las siguientes integrales usaremos el comando \texttt{quadgk}.

\[
\displaystyle I_1 = \int_{0}^{\pi/3} \! \frac{1}{1 - \sin(x)} \, dx
\]

\begin{table}[h]
    \centering
    \rowcolors{2}{gray!10}{white}
    \begin{tabular}{|c|c|c|}
        \hline
        \rowcolor{gray!30}
         & Gauss Lobatto 9 nodos & quadgk \\
        \hline
         errores & 2.732050857019763 & 2.732050807568876 \\
        \hline
    \end{tabular}
\end{table}

\vspace{0.5em}

\begin{table}[h]
    \centering
    \rowcolors{2}{gray!10}{white}
    \begin{tabular}{|c|}
        \hline
        \rowcolor{gray!30}
         error absoluto \\
        \hline
         4.945088694086053e-08 \\
        \hline
    \end{tabular}
\end{table}
\end{frame}


%------------------------------------------------
%------------------------------------------------

\begin{frame} 
\frametitle{Aproximación mediante Gauss-Lobatto con 9 nodos}

\[
\displaystyle I_2 = \int_{0}^{\pi} \! \frac{\sin(x)}{x^{3/2}} \, dx
\]

Para \( I_2 \), se utilizó la función \texttt{integraGaussLobattoImpropio} debido a la singularidad de la función \( f(x) = \frac{\sin(x)}{x^{3/2}} \) en \( x = 0 \).

\vspace{1em}

\begin{table}[h]
    \centering
    \rowcolors{2}{gray!10}{white}
    \begin{tabular}{|c|c|c|}
        \hline
        \rowcolor{gray!30}
         & Gauss Lobatto 9 nodos & quadgk \\
        \hline
         errores & 2.651469905786794 & 2.651469253040811 \\
        \hline
    \end{tabular}
\end{table}

\vspace{0.5em}

\begin{table}[h]
    \centering
    \rowcolors{2}{gray!10}{white}
    \begin{tabular}{|c|}
        \hline
        \rowcolor{gray!30}
         error absoluto \\
        \hline
         6.527459821548121e-07 \\
        \hline
    \end{tabular}
\end{table}
\end{frame}


%------------------------------------------------
%------------------------------------------------

\begin{frame}
\frametitle{Aproximación mediante Gauss-Lobatto con 9 nodos}

\[
\displaystyle I_3 = \int_{0}^{1} \! \frac{e^x}{x^2} \, dx
\]

\vspace{1em}

En el caso de \( I_3 \), tanto la cuadratura de Gauss-Lobatto como \texttt{quadgk} devuelven un valor de \( \infty \).  
Esto indica que la integral \textbf{diverge}, lo cual es esperado debido a la \textbf{singularidad} de la función  
\( f(x) = \frac{e^x}{x^2} \) en \( x = 0 \).

\vspace{0.5em}

La función crece de forma extremadamente rápida cerca de \( x = 0 \), por lo que la integral no converge en el intervalo \( [0, 1] \).

\vspace{0.5em}

Este resultado confirma que \( I_3 \) es una \textbf{integral impropia divergente} que no tiene un valor finito.
\end{frame}



%------------------------------------------------
\section{Comparación con Gauss-Legendre}
%------------------------------------------------
\begin{frame}
\frametitle{Comparación con Gauss-Legendre}
\displaystyle I_1=\int_{0}^{\pi/3} \! \frac{1}{1-\sin(x)}  \,dx

\begin{table}[h]

    \rowcolors{2}{gray!10}{white} % filas alternadas en gris claro
    \begin{tabular}{|c|c|c|c|}
        \hline
        \rowcolor{gray!30} % color de la primera fila
         & I_1 & errores\\
        \hline
         Gauss-Lobatto &  2.723600154858061 & 4.759629890216388e-08\\
        \hline
        Gauss-Legendre &  2.723600103938956 & 3.322805586236655e-09\\
        \hline
        quadgk & 2.723600107261762 &\\
        \hline
    \end{tabular}
    
    \label{tab:my_label}
\end{table}

\end{frame}

\begin{frame}
\frametitle{Comparación con Gauss-Legendre}
\displaystyle I_2=\int_{0}^{\pi} \! \frac{\sin(x)}{x^{3/2}}  \,dx

\begin{table}[h]

    \rowcolors{2}{gray!10}{white} % filas alternadas en gris claro
    \begin{tabular}{|c|c|c|c|}
        \hline
        \rowcolor{gray!30} % color de la primera fila
         & I_2 & errores\\
        \hline
         Gauss-Lobatto &  3.564891079137420 & 0.976667467013740\\
        \hline
        Gauss-Legendre &  2.480811316956006 & 0.107412295167674\\
        \hline
        quadgk & 2.588223612123681 &\\
        \hline
    \end{tabular}
    
    \label{tab:my_label}
\end{table}

\end{frame}

\begin{frame}
\frametitle{Comparación con Gauss-Legendre}
\(\displaystyle I_3=\int_{0}^{1} \! \frac{e^x}{x^2}  \,dx\)

\begin{table}[h]

    \rowcolors{2}{gray!10}{white} % filas alternadas en gris claro
    \begin{tabular}{|c|c|c|c|}
        \hline
        \rowcolor{gray!30} % color de la primera fila
         & I_3 & errores\\
        \hline
         Gauss-Lobatto &  1.391952022433983e+4 & 1.291301606846351e+4\\
        \hline
        Gauss-Legendre &  1.676112081728000e+2 & 8.388929477035150e+2\\
        \hline
        quadgk & 1.006504155876315e+3 &\\
        \hline
    \end{tabular}
    
    \label{tab:my_label}
\end{table}

\end{frame}
%------------------------------------------------
%% Bibliografía
\begin{frame}
\frametitle{Referencias}
\footnotesize{
  \begin{thebibliography}{99} 
    \bibitem[Smith, 2012]{p1} John Smith (2012)
      \newblock Title of the publication
      \newblock \emph{Journal Name} 12(3), 45 -- 678.
  \end{thebibliography}
}
\end{frame}

%------------------------------------------------

\begin{frame}
\Huge{\centerline{Fin.}}
\end{frame}

%----------------------------------------------------------------------------------------

\end{document} 